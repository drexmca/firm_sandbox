\documentclass[letterpaper,12pt]{article}

\usepackage{threeparttable}
\usepackage{geometry}
\geometry{letterpaper,tmargin=1in,bmargin=1in,lmargin=1.25in,rmargin=1.25in}
\usepackage[format=hang,font=normalsize,labelfont=bf]{caption}
\usepackage{amsmath}
\usepackage{multirow}
\usepackage{array}
\usepackage{delarray}
\usepackage{amssymb}
\usepackage{amsthm}
\usepackage{lscape}
\usepackage{natbib}
\usepackage{setspace}
\usepackage{float,color}
\usepackage[pdftex]{graphicx}
\usepackage{pdfsync}
\usepackage{verbatim}
\usepackage{placeins}
\usepackage{geometry}
\usepackage{pdflscape}
\synctex=1
\usepackage{hyperref}
\hypersetup{colorlinks,linkcolor=red,urlcolor=blue,citecolor=red}
\usepackage{bm}


\theoremstyle{definition}
\newtheorem{theorem}{Theorem}
\newtheorem{acknowledgement}[theorem]{Acknowledgement}
\newtheorem{algorithm}[theorem]{Algorithm}
\newtheorem{axiom}[theorem]{Axiom}
\newtheorem{case}[theorem]{Case}
\newtheorem{claim}[theorem]{Claim}
\newtheorem{conclusion}[theorem]{Conclusion}
\newtheorem{condition}[theorem]{Condition}
\newtheorem{conjecture}[theorem]{Conjecture}
\newtheorem{corollary}[theorem]{Corollary}
\newtheorem{criterion}[theorem]{Criterion}
\newtheorem{definition}{Definition} % Number definitions on their own
\newtheorem{derivation}{Derivation} % Number derivations on their own
\newtheorem{example}[theorem]{Example}
\newtheorem{exercise}[theorem]{Exercise}
\newtheorem{lemma}[theorem]{Lemma}
\newtheorem{notation}[theorem]{Notation}
\newtheorem{problem}[theorem]{Problem}
\newtheorem{proposition}{Proposition} % Number propositions on their own
\newtheorem{remark}[theorem]{Remark}
\newtheorem{solution}[theorem]{Solution}
\newtheorem{summary}[theorem]{Summary}
\bibliographystyle{aer}
\newcommand\ve{\epsilon}
%\renewcommand\theenumi{\roman{enumi}}
\newcommand\norm[1]{\left\lVert#1\right\rVert}

\begin{document}

\title{Algorithms for solving for SS and Time Path of OG model with heterogeneous static firms}
\date{\today}
\author{}
\maketitle

\begin{spacing}{1.5}
\pagenumbering{arabic}


The model assumed for these solutions is one without taxes or bequests.  Endogenous labor.  $J$ ability types.  $M$ firms.  $I$ consumption goods.  $T$ periods from initial state to the steady state.

\section{Steady State}

\begin{enumerate}
\item Guess $\bar{r}$ and $\bar{w}$
\item Determine output prices, $\bar{p}_{m}$ using zero profit condition and firm FOCS together. These imply:

\begin{equation}
\bar{p}_{m}  = \left[(1-\gamma_{m})\left(\frac{\bar{w}}{\bar{A}_{m}}\right)^{1-\epsilon_{m}} + \gamma_{m}\left(\frac{(\bar{r}+\delta_{m})}{\bar{A}_{m}}\right)^{1-\epsilon_{m}} \right]^{\frac{1}{1-\epsilon_{m}}}
\end{equation}
\item Determine consumption goods prices using the fixed coefficient matrix $\Pi$ mapping output goods to consumption goods: $\bar{p}^{c}_{i} = \sum_{m=1}^{M}\pi_{i,m}\bar{p}_{m}$
\item Determine the price of the composite consumption good: $\tilde{p} =  \prod_{i=1}^{I}\left( \frac{\bar{p}^{c}_{i}}{\alpha_{i}} \right)^{\alpha_{i}}$
\item With prices $\bar{r}, \bar{w}, \bar{p}^{c}_{i}, \tilde{p}$, we can solve the HH problem.  We can solve the problem for each type $j$ separately using a root-finding algorithm to solve for the $2\times S$ unknowns from $2\times S$ equations: the HH FOCs for the choices of savings and labor supply in each year of life.  This yields $\bar{b}_{j,s}$ and $\bar{n}_{j,s}$.
\item Using the HH's budget constraint and the vector of prices, we can solve for composite consumption, $\tilde{c}_{j,s}$.
\item We can solve for the total demand for each consumption good using the necessary conditions from the HH's subutility function.  These imply (where $\bar{c}_{i}$ is the minimum expenditure on good $i$): 

\begin{equation}
\bar{c}_{i,j,s}  = \frac{\alpha_{i} \tilde{p}_{s}\tilde{c}_{j,s}}{\bar{p}^{c}_{i}} + \bar{c}_{i},
\end{equation}

\item Summing over $J$ and $S$ we can get total demand for consumption of good $i$: $\bar{C}_{i} = \sum_{j=1}^{J}\sum_{s=1}^{S} \bar{c}_{i,j,s}$
\item With $\bar{C}_{i}$ and prices, we can solve for the demand for output from each industry using the resource constraint, the fixed coefficient matrices $\Pi$ and $\Xi$ (where $\Xi$ is the fixed coefficient input-output matrix - mapping production goods to capital goods), and the demand for investment.
	\begin{itemize}
	\item Investment demand for industry $m$ in the SS is given by: $\bar{I}_{m} = \delta_{m}\bar{K}_{m}$
	\item Demand for industry $m$ output from consumption is given by: $\bar{X}^{c}_{m} = \sum_{i=1}^{I} \pi_{i,m}\bar{C}_{i}$
	\item Demand for industry $m$ output from investment demand is given by: $\bar{X}^{i}_{m} =  \sum_{j=1}^{M} \xi_{j,m}\bar{I}_{j} = \sum_{j=1}^{M} \xi_{j,m}\delta_{j}\bar{K}_{j} $
	\item And we can use the firm's FOC for capital, labor, and the production function together,  to write the demand for capital as a function of output and factor prices: 
	
	\begin{equation}
	\label{eqn:k_demand2}
	\bar{K}_{m} = \frac{\bar{X}_{m}}{\bar{A}_{m}}\left[\gamma_{m}^{\frac{1}{\epsilon_{m}}}+(1-\gamma_{m})^{\frac{1}{\epsilon_{m}}}\left(\frac{\bar{r}+\delta_{m}}{\bar{w}}\right)^{\epsilon_{m}-1}\left(\frac{1-\gamma_{m}}{\gamma_{m}}	\right)^{\frac{\epsilon_{m}-1}{\epsilon_{m}}}\right]^{\frac{\epsilon_{m}}{1-\epsilon_{m}}}
	\end{equation}
	
	\item Resource constraint for industry $m$: 
	\begin{equation}
	\begin{split}
	\bar{X}_{m} &= \bar{X}^{c}_{m} + \bar{X}^{i}_{m} \\
	&= \sum_{i=1}^{I} \pi_{i,m}\bar{C}_{i} +\sum_{j=1}^{M} \xi_{j,m}\bar{I}_{j} \\
	 &= \sum_{i=1}^{I} \pi_{i,m}\bar{C}_{i} +\sum_{j=1}^{M} \xi_{j,m}\delta_{j}\bar{K}_{j}  \\
	 &= \sum_{i=1}^{I} \pi_{i,m}\bar{C}_{i} +\sum_{j=1}^{M} \xi_{j,m}\delta_{j} \frac{\bar{X}_{j}}{\bar{A}_{j}}\left[\gamma_{j}^{\frac{1}{\epsilon_{j}}}+(1-\gamma_{j})^{\frac{1}{\epsilon_{j}}}\left(\frac{\bar{r}+\delta_{j}}{\bar{w}}\right)^{\epsilon_{j}-1}\left(\frac{1-\gamma_{j}}{\gamma_{j}}	\right)^{\frac{\epsilon_{j}-1}{\epsilon_{j}}}\right]^{\frac{\epsilon_{j}}{1-\epsilon_{j}}}
	\end{split}
	\end{equation}
	
	\item The above is a system of $M$ equations and $M$ unknowns: a root finder can be used to solve for $\bar{X}_{m}$
	\end{itemize}

\item With $\bar{X}_{m}$ and prices, we can solve for the demand for capital from each industry.  We'll use the capital market clearing condition ($\sum_{m=1}^{M} \bar{K}_{m}=\sum_{j=1}^{J}\sum_{s=1}^{S} \bar{b}_{j,s}$) and the firm's FOC for capital together to find these demands.  Denoting the supply of capital as $\bar{K}^{s}=\sum_{j=1}^{J}\sum_{s=1}^{S} \bar{b}_{j,s}$, we have:

		\begin{equation}
	\begin{split}
	& \bar{r}+\delta_{m} = \bar{p}_{m}\left(\frac{\gamma_{m}\bar{X}_{m}}{\bar{K}_{m}}\right)^{\frac{1}{\epsilon_{m}}}\bar{A}_{m}^{\frac{\epsilon_{m}-1}{\epsilon_{m}}} \\
	\implies & \frac{\left(\frac{\bar{r}+\delta_{m}}{\bar{p}_{m}}\bar{A}_{m}^{\frac{1-\epsilon_{m}}{\epsilon_{m}}}\right)^{\epsilon_{m}}}{\gamma_{m}} = \frac{\bar{X}_{m}}{\bar{K}_{m}} \\
	\implies &  \bar{K}_{m} = \gamma_{m}\bar{X}_{m} \left(\frac{\bar{r}+\delta_{m}}{\bar{p}_{m}}\bar{A}_{m}^{\frac{1-\epsilon_{m}}{\epsilon_{m}}}\right)^{-\epsilon_{m}} \\
	& \text{Summing both sides over $M$ yields:} \\
	\implies &  \underbrace{\sum_{m=1}^{M} \bar{K}_{m}= \bar{K}^{s}}_{\text{By capital market clearing}} =\sum_{m=1}^{M} \gamma_{m}\bar{X}_{m} \left(\frac{\bar{r}+\delta_{m}}{\bar{p}_{m}}\bar{A}_{m}^{\frac{1-\epsilon_{m}}{\epsilon_{m}}}\right)^{-\epsilon_{m}} \\
	& \text{We can rearrange this equation to solve for $\bar{K}_{m}$ as:}\\
	\implies & \bar{K}_{m} = \bar{K}^{s} -\sum_{k \neq=m} \gamma_{k}\bar{X}_{k} \left(\frac{\bar{r}+\delta_{k}}{\bar{p}_{k}}\bar{A}_{k}^{\frac{1-\epsilon_{k}}{\epsilon_{k}}}\right)^{-\epsilon_{k}}\\
	& \text{Note that the above only holds for the correct $\bar{r}$.  }\\
	& \text{To make sure it always holds (even outside eq'm) we'll substitute for $\bar{r}$} \\
	& \text{using FOC for capital:} \\
	\implies & \bar{K}_{m} = \bar{K}^{s} -\sum_{k \neq=m} \gamma_{k}\bar{X}_{k} \left(\frac{\overbrace{\bar{p}_{i}\left(\frac{\gamma_{i}\bar{X}_{i}}{\bar{K}_{i}}\right)^{\frac{1}{\epsilon_{i}}}A_{i,t}^{\frac{\epsilon_{i}-1}{\epsilon_{i}}}-\delta_{i}}^{=\bar{r}, \ \forall i}+\delta_{k}}{\bar{p}_{k}}\bar{A}_{k}^{\frac{1-\epsilon_{k}}{\epsilon_{k}}}\right)^{-\epsilon_{k}}\\
	\end{split}
	\end{equation}

%	\begin{equation}
%	\bar{K}_{m} = \bar{K}^{s} - \sum_{k \neq m} \gamma_{k}\bar{X}_{k}\left(\frac{\bar{p}_{m}\left(\frac{\gamma_{m}\bar{X}_{m}}{\bar{K}_{m}}\right)\bar{A}_{m}^{\frac{\epsilon_{m}-1}{\epsilon_{m}}}-\delta_{m}+\delta_{k}}{\bar{p}_{k}}\bar{A}_{k}^{\frac{1-\epsilon_{k}}{\epsilon_{k}}}\right)^{-\epsilon_{k}}
%	\end{equation}

	This is a system of $M$ nonlinear equations that can be solved for $\bar{K}_{m}$ using a root finding algorithm.  Note that the $i$ in the numerator within the parentheses on the right hand side of the equation can be any $i$, but it doesn't work if $i=k$.  The capital demands that result from this are consistent with market clearing and insure that the implied interest rate is the same across firms.  However, they these demands do not necessarily mean that the interest rate implied from these capital demands is the same interest rate that gave rise to the supply of capital used in the solution.  That will only be the case for the equilibrium value of $\bar{r}$.  
	
\item With $\bar{X}_{m}$ and prices, we can solve for the demand for labor from each industry.  We'll use the labor market clearing condition ($\sum_{m=1}^{M} \bar{EL}_{m}=\sum_{j=1}^{J}\sum_{s=1}^{S}e_{j,s} \bar{n}_{j,s}$) and the firm's FOC for labor together to find these demands.  Denoting the supply of labor as $\bar{L}^{s}=\sum_{j=1}^{J}\sum_{s=1}^{S} e_{j,s}\bar{n}_{j,s}$, we have:

	\begin{equation}
	\begin{split}
	& \bar{w} = \bar{p}_{m}\left(\frac{(1-\gamma_{m})\bar{X}_{m}}{\overline{EL}_{m}}\right)^{\frac{1}{\epsilon_{m}}}\bar{A}_{m}^{\frac{\epsilon_{m}-1}{\epsilon_{m}}} \\
	\implies & \frac{\left(\frac{\bar{w}}{\bar{p}_{m}}\bar{A}_{m}^{\frac{1-\epsilon_{m}}{\epsilon_{m}}}\right)^{\epsilon_{m}}}{(1-\gamma_{m})} = \frac{\bar{X}_{m}}{\overline{EL}_{m}} \\
	\implies &  \overline{EL}_{m} = (1-\gamma_{m})\bar{X}_{m} \left(\frac{\bar{w}}{\bar{p}_{m}}\bar{A}_{m}^{\frac{1-\epsilon_{m}}{\epsilon_{m}}}\right)^{-\epsilon_{m}} \\
	& \text{Summing both sides over $M$ yields:} \\
	\implies &  \underbrace{\sum_{m=1}^{M} \overline{EL}_{m}= \bar{L}^{s}}_{\text{By labor market clearing}} =\sum_{m=1}^{M} (1-\gamma_{m})\bar{X}_{m} \left(\frac{\bar{w}}{\bar{p}_{m}}\bar{A}_{m}^{\frac{1-\epsilon_{m}}{\epsilon_{m}}}\right)^{-\epsilon_{m}} \\
	& \text{We can rearrange this equation to solve for $\overline{EL}_{m}$ as:}\\
	\implies & \overline{EL}_{m} = \bar{L}^{s} -\sum_{k \neq=m} (1- \gamma_{k})\bar{X}_{k} \left(\frac{\bar{w}}{\bar{p}_{k}}\bar{A}_{k}^{\frac{1-\epsilon_{k}}{\epsilon_{k}}}\right)^{-\epsilon_{k}}\\
	& \text{Note that the above only holds for the correct $\bar{w}$.  }\\
	& \text{To make sure it always holds (even outside eq'm) we'll substitute for $\bar{w}$} \\
	& \text{using FOC for labor:} \\
	\implies & \overline{EL}_{m} = \bar{L}^{s} -\sum_{k \neq=m} (1- \gamma_{k})\bar{X}_{k} \left(\frac{\overbrace{\bar{p}_{i}\left(\frac{(1-\gamma_{i})\bar{X}_{i}}{\overline{EL}_{i}}\right)^{\frac{1}{\epsilon_{i}}}A_{i,t}^{\frac{\epsilon_{i}-1}{\epsilon_{i}}}}^{=\bar{w}, \ \forall i}}{\bar{p}_{k}}\bar{A}_{k}^{\frac{1-\epsilon_{k}}{\epsilon_{k}}}\right)^{-\epsilon_{k}}\\
	\end{split}
	\end{equation}

%	\begin{equation}
%	\bar{EL}_{m} = \bar{L}^{s} - \sum_{k\neq m} (1- \gamma_{k})\bar{X}_{k}\left(\frac{\bar{p}_{m}\left(\frac{(1-\gamma_{m})\bar{X}_{m}}{\bar{EL}_{m}}\right)\bar{A}_{m}^{\frac{\epsilon_{m}-1}{\epsilon_{m}}}-\delta_{m}+\delta_{k}}{\bar{p}_{k}}\bar{A}_{k}^{\frac{1-\epsilon_{k}}{\epsilon_{k}}}\right)^{-\epsilon_{k}}
%	\end{equation}
	This is a system of $M$ nonlinear equations that can be solved for $\bar{EL}_{m}$ using a root finding algorithm.
	
\item Use the demands for capital and labor found above in the FOC for a particular firm $m$ (can be any firm), this will imply and interest rate and wage rate.  Call these $r_{new}$ and $w_{new}$, respectively.  We have:

	\begin{equation}
	r_{new} = \bar{p}_{m}\left(\frac{\gamma_{m}\bar{X}_{m}}{\bar{K}_{m}}\right)\bar{A}_{m}^{\frac{\epsilon_{m}-1}{\epsilon_{m}}}-\delta_{m}, \forall \ m
	\end{equation}

	\begin{equation}
	w_{new} = \bar{p}_{m}\left(\frac{(1-\gamma_{m})\bar{X}_{m}}{\bar{EL}_{m}}\right)\bar{A}_{m}^{\frac{\epsilon_{m}-1}{\epsilon_{m}}}, \forall \ m
	\end{equation}

\item If $r_{new}$ and $w_{new}$ are equal to $\bar{r}$ and $\bar{w}$ guessed, then stop.  Else, update the guesses over $\bar{r}$ and $\bar{w}$ and repeat.  One can use $r_{new}$ and $w_{new}$ to inform the new guess.

\end{enumerate}



\section{Time Path}

\begin{enumerate}
\item Guess the time paths for factor prices: $r_{t}$ and $w_{t}$
\item Determine time path for output prices, $p_{m,t}$ using zero profit condition and firm FOCS together. These imply:

\begin{equation}
p_{m,t}  = \left[(1-\gamma_{m})\left(\frac{w_{t}}{A_{m,t}}\right)^{1-\epsilon_{m}} + \gamma_{m}\left(\frac{(r_{t}+\delta_{m})}{A_{m,t}}\right)^{1-\epsilon_{m}} \right]^{\frac{1}{1-\epsilon_{m}}}
\end{equation}
\item Determine the time path for consumption goods prices using the fixed coefficient matrix $\Pi$ mapping output goods to consumption goods: $p^{c}_{i,t} = \sum_{m=1}^{M}\pi_{i,m}p_{m,t}$
\item Determine time path for the price of the composite consumption good: $\tilde{p}_{t} =  \prod_{i=1}^{I}\left( \frac{p^{c}_{i,t}}{\alpha_{i}} \right)^{\alpha_{i}}$
\item With the time paths of prices $r_{t}, w_{t}, p^{c}_{i,t}, \tilde{p}_{t}$, we can solve the HH problem.  I think we can solve the path for each type $j$ and cohort separately using a root-finding algorithm to solve for the $2\times S$ unknowns from $2\times S$ equations: the HH FOCs for the choices of savings and labor supply.  This yields, $b_{j,s,t}$ and $n_{j,s,t}$.
\item Using the HH's budget constraint and the vector of prices, we can solve for composite consumption, $\tilde{c}_{j,s,t}$.
\item We can solve for the total demand for each consumption good using the necessary conditions from the HH's subutility function.  These imply (where $\bar{c}_{i}$ is the minimum expenditure on good $i$): 

\begin{equation}
c_{i,j,s,t}  = \frac{\alpha_{i} \tilde{p}_{s,t}\tilde{c}_{j,s,t}}{p_{i,t}} + \bar{c}_{i},
\end{equation}

\item Summing over $J$ and $S$ we can get time path for the demand for consumption of good $i$: $C_{i,t} = \sum_{j=1}^{J}\sum_{s=1}^{S} c_{i,j,s,t}$
\item With $C_{i,t}$ and prices, we can solve for the demand for output from each industry using the resource constraint, the fixed coefficient matrices $\Pi$ and $\Xi$ (where $\Xi$ is the fixed coefficient input-output matrix - mapping production goods to capital goods), and the demand for investment.
	\begin{itemize}
	\item Investment demand for industry $m$ in period $t$ is given by: $I_{m,t} = K_{m,t+1} - (1 - \delta_{m})K_{m,t}$
	\item Demand for industry $m$ output in period $t$ from consumption is given by: $X^{c}_{m,t} = \sum_{i=1}^{I} \pi_{i,m}C_{i,t}$
	\item Demand for industry $m$ output in period $t$ from investment demand is given by: $X^{i}_{m,t} =  \sum_{j=1}^{M} \xi_{j,m}I_{j,t} = \sum_{j=1}^{M} \xi_{j,m}\left(K_{j,t+1} - (1\delta_{m})K_{j,t}\right) $
	\item And we can use the firm's FOC for capital, labor, and the production function together,  to write the demand for capital in each period as a function of output and factor prices: 
	
	\begin{equation}
	\label{eqn:k_demand2}
	K_{m,t} = \frac{X_{m,t}}{A_{m,t}}\left[\gamma_{m}^{\frac{1}{\epsilon_{m}}}+(1-\gamma_{m})^{\frac{1}{\epsilon_{m}}}\left(\frac{r_{t}+\delta_{m}}{w_{t}}\right)^{\epsilon_{m}-1}\left(\frac{1-\gamma_{m}}{\gamma_{m}}	\right)^{\frac{\epsilon_{m}-1}{\epsilon_{m}}}\right]^{\frac{\epsilon_{m}}{1-\epsilon_{m}}}
	\end{equation}
	
	\item Resource constraint for industry $m$ in year $t$: 
	\begin{equation}
	\begin{split}
	X_{m,t} &= X^{c}_{m,t} + X^{i}_{m,t} \\
	&= \sum_{i=1}^{I} \pi_{i,m}C_{i,t} +\sum_{j=1}^{M} \xi_{j,m}I_{j,t} \\
	 &= \sum_{i=1}^{I} \pi_{i,m}C_{i,t} +\sum_{j=1}^{M} \xi_{j,m}\left(K_{j,t+1} - (1-\delta_{j})K_{j,t} \right)  \\
	 &= \sum_{i=1}^{I} \pi_{i,m}C_{i,t} ... \\
	 & \quad +\sum_{j=1}^{M} \xi_{j,m}\left[\left( \frac{X_{j,t+1}}{A_{j,t+1}}\left[\gamma_{j}^{\frac{1}{\epsilon_{j}}}+(1-\gamma_{j})^{\frac{1}{\epsilon_{j}}}\left(\frac{r_{t+1}+\delta_{j}}{w_{t+1}}\right)^{\epsilon_{j}-1}\left(\frac{1-\gamma_{j}}{\gamma_{j}}\right)^{\frac{\epsilon_{j}-1}{\epsilon_{j}}}\right]^{\frac{\epsilon_{j}}{1-\epsilon_{j}}}\right) ... \right. \\
	 & \left. \quad - (1-\delta_{j})\left( \frac{X_{j,t}}{A_{j,t}}\left[\gamma_{j}^{\frac{1}{\epsilon_{j}}}+(1-\gamma_{j})^{\frac{1}{\epsilon_{j}}}\left(\frac{r_{t}+\delta_{j}}{w_{t}}\right)^{\epsilon_{j}-1}\left(\frac{1-\gamma_{j}}{\gamma_{j}}	\right)^{\frac{\epsilon_{j}-1}{\epsilon_{j}}}\right]^{\frac{\epsilon_{j}}{1-\epsilon_{j}}}\right) \right]
	\end{split}
	\end{equation}
	
	\item The above is a system of $M\times T$ equations and $M\times T$ unknowns: a root finder can be used to solve for $X_{m,t}$
	\item Alternatively, one can leverage the recursive nature of the problem.  In particular, since we know the steady state solution, we have $K_{m,T}$ for all $m$.  Thus we can find $I_{m,T-1} = K_{m,T} - (1-\delta_{m})K_{m,T-1}$.  Then using the equations above, we can solve a system of $M$ equations and $M$ unknowns to find $X_{m,T-1}$.  We then repeat this for each period from $T-1$ back to the initial period.  This yields the time path of output demand, $X_{m,t}$ without having to solve a very large system of equations at once.
	\end{itemize}

\item With the time path for output demand, $X_{m,t}$, and prices, we can solve for the demand for capital from each industry in each period $t$.  We'll use the capital market clearing condition ($\sum_{m=1}^{M} K_{m,t}=\sum_{j=1}^{J}\sum_{s=1}^{S} b_{j,s,t}$) and the firm's FOC for capital together to find these demands.  Denoting the supply of capital as $K^{s}_{t}=\sum_{j=1}^{J}\sum_{s=1}^{S} b_{j,s,t}$, we can start with the firm's FOC for its choice of capital:

	\begin{equation}
	\begin{split}
	& r_{t}+\delta_{m} = p_{m,t}\left(\frac{\gamma_{m}X_{m,t}}{K_{m,t}}\right)^{\frac{1}{\epsilon_{m}}}A_{m,t}^{\frac{\epsilon_{m}-1}{\epsilon_{m}}} \\
	\implies & \frac{\left(\frac{r_{t}+\delta_{m}}{p_{m,t}}A_{m,t}^{\frac{1-\epsilon_{m}}{\epsilon_{m}}}\right)^{\epsilon_{m}}}{\gamma_{m}} = \frac{X_{m,t}}{K_{m,t}} \\
	\implies &  K_{m,t} = \gamma_{m}X_{m,t} \left(\frac{r_{t}+\delta_{m}}{p_{m,t}}A_{m,t}^{\frac{1-\epsilon_{m}}{\epsilon_{m}}}\right)^{-\epsilon_{m}} \\
	& \text{Summing both sides over $M$ yields:} \\
	\implies &  \underbrace{\sum_{m=1}^{M} K_{m,t}= K^{s}_{t}}_{\text{By capital market clearing}} =\sum_{m=1}^{M} \gamma_{m}X_{m,t} \left(\frac{r_{t}+\delta_{m}}{p_{m,t}}A_{m,t}^{\frac{1-\epsilon_{m}}{\epsilon_{m}}}\right)^{-\epsilon_{m}} \\
	& \text{We can rearrange this equation to solve for $K_{m,t}$ as:}\\
	\implies & K_{m,t} = K^{s}_{t} -\sum_{k \neq=m} \gamma_{k}X_{k,t} \left(\frac{r_{t}+\delta_{k}}{p_{k,t}}A_{k,t}^{\frac{1-\epsilon_{k}}{\epsilon_{k}}}\right)^{-\epsilon_{k}}\\
	& \text{Note that the above only holds for the correct $r_{t}$.  }\\
	& \text{To make sure it always holds (even outside eq'm) we'll substitute for $r_{t}$} \\
	& \text{using FOC for capital:} \\
	\implies & K_{m,t} = K^{s}_{t} -\sum_{k \neq=m} \gamma_{k}X_{k,t} \left(\frac{\overbrace{p_{i,t}\left(\frac{\gamma_{i}X_{i,t}}{K_{i,t}}\right)^{\frac{1}{\epsilon_{i}}}A_{i,t}^{\frac{\epsilon_{i}-1}{\epsilon_{i}}}-\delta_{i}}^{=r_{t}, \ \forall i}+\delta_{k}}{p_{k,t}}A_{k,t}^{\frac{1-\epsilon_{k}}{\epsilon_{k}}}\right)^{-\epsilon_{k}}\\
	\end{split}
	\end{equation}
	
	For each year, $t$, this is a system of $M$ nonlinear equations that can be solved for $K_{m,t}$ using a root finding algorithm.
	
\item With the time path for output demand, $X_{m,t}$ and prices, we can solve for the demand for labor from each industry.  We'll use the labor market clearing condition ($\sum_{m=1}^{M} EL_{m,t}=\sum_{j=1}^{J}\sum_{s=1}^{S}e_{j,s} n_{j,s,t}$) and the firm's FOC for labor together to find these demands.  Denoting the supply of labor as $L^{s,t}=\sum_{j=1}^{J}\sum_{s=1}^{S} e_{j,s}n_{j,s,t}$, we have:


	\begin{equation}
	\begin{split}
	& w_{t} = p_{m,t}\left(\frac{(1-\gamma_{m})X_{m,t}}{EL_{m,t}}\right)^{\frac{1}{\epsilon_{m}}}A_{m,t}^{\frac{\epsilon_{m}-1}{\epsilon_{m}}} \\
	\implies & \frac{\left(\frac{w_{t}}{p_{m,t}}A_{m,t}^{\frac{1-\epsilon_{m}}{\epsilon_{m}}}\right)^{\epsilon_{m}}}{(1-\gamma_{m})} = \frac{X_{m,t}}{EL_{m,t}} \\
	\implies &  EL_{m,t} =(1- \gamma_{m})X_{m,t} \left(\frac{w_{t}}{p_{m,t}}A_{m,t}^{\frac{1-\epsilon_{m}}{\epsilon_{m}}}\right)^{-\epsilon_{m}} \\
	& \text{Summing both sides over $M$ yields:} \\
	\implies &  \underbrace{\sum_{m=1}^{M} EL_{m,t}= L^{s}_{t}}_{\text{By labor market clearing}} =\sum_{m=1}^{M} (1- \gamma_{m})X_{m,t} \left(\frac{w_{t}}{p_{m,t}}A_{m,t}^{\frac{1-\epsilon_{m}}{\epsilon_{m}}}\right)^{-\epsilon_{m}} \\
	& \text{We can rearrange this equation to solve for $EL_{m,t}$ as:}\\
	\implies & EL_{m,t} = L^{s}_{t} -\sum_{k \neq=m} (1-\gamma_{k})X_{k,t} \left(\frac{w_{t}}{p_{k,t}}A_{k,t}^{\frac{1-\epsilon_{k}}{\epsilon_{k}}}\right)^{-\epsilon_{k}}\\
	& \text{Note that the above only holds for the correct $w_{t}$.  }\\
	& \text{To make sure it always holds (even outside eq'm) we'll substitute for $w_{t}$} \\
	& \text{using FOC for labor:} \\
	\implies & EL_{m,t} = L^{s}_{t} -\sum_{k \neq=m} (1- \gamma_{k})X_{k,t} \left(\frac{\overbrace{p_{i,t}\left(\frac{(1-\gamma_{i})X_{i,t}}{EL_{i,t}}\right)^{\frac{1}{\epsilon_{i}}}A_{i,t}^{\frac{\epsilon_{i}-1}{\epsilon_{i}}}}^{=w_{t}, \ \forall i}}{p_{k,t}}A_{k,t}^{\frac{1-\epsilon_{k}}{\epsilon_{k}}}\right)^{-\epsilon_{k}}\\
	\end{split}
	\end{equation}

%	\begin{equation}
%	EL_{m,t} = L^{s,t} - \sum_{k\neq m} (1- \gamma_{k})X_{k,t}\left(\frac{p_{m,t}\left(\frac{(1-\gamma_{m})X_{m,t}}{EL_{m,t}}\right)A_{m,t}^{\frac{\epsilon_{m}-1}{\epsilon_{m}}}-\delta_{m}+\delta_{k}}{p_{k,t}}A_{k,t}^{\frac{1-\epsilon_{k}}{\epsilon_{k}}}\right)^{-\epsilon_{k}}
%	\end{equation}
	For each year, $t$, this is a system of $M$ nonlinear equations that can be solved for $EL_{m,t}$ using a root finding algorithm.
	
\item Use the demands for capital and labor found above in the FOC for a particular firm, this will imply and interest rate and wage rate in each period, $t$.  Call these $r_{new,t}$ and $w_{new,t}$, respectively.  We have:

	\begin{equation}
	r_{new,t} = p_{m}\left(\frac{\gamma_{m}X_{m,t}}{K_{m,t}}\right)A_{m,t}^{\frac{\epsilon_{m}-1}{\epsilon_{m}}}-\delta_{m}, \forall \ m, t
	\end{equation}

	\begin{equation}
	w_{new} = p_{m,t}\left(\frac{(1-\gamma_{m})X_{m,t}}{EL_{m,t}}\right)A_{m,t}^{\frac{\epsilon_{m}-1}{\epsilon_{m}}}, \forall \ m, t
	\end{equation}

\item If $r_{new,t}$ and $w_{new,t}$ are equal the time paths $r_{t}$ and $w_{t}$ guessed, then stop.  Else, update the guesses over $r_{t}$ and $w_{t}$ and repeat.  One can use $r_{new,t}$ and $w_{new,t}$ to inform the new guess. 

\end{enumerate}

\end{spacing}
\end{document}